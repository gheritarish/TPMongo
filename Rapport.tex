\documentclass{Cours}

\title{TP 3 -- noSQL}
\author{Alexandre Lepelletier et Baptiste Marlet}

\begin{document}
\maketitle
\tableofcontents

\newpage
\section{Introduction}
L'objectif de ce TP est de mettre en place une base de données \texttt{MongoDB} afin de prendre en main le logiciel. Nous allons faire quelques requêtes pour intégrer des données puis nous allons chercher à faire quelques requêtes.

\newpage
\section{Partie \texttt{MongoDB}}
Dans un premier temps, nous avons créé notre base de données, puis nous avons mis dans cette base les différentes données. Pour cela, nous avons écrit un script afin d'insérer les différentes données dans la base.

Puis, nous avons utilisé un script \texttt{Python} basique pour afficher les différentes fiches, avant de répondre aux questions. Cela nous a permis d'utiliser la fonction \texttt{count} de \texttt{pymongo}. D'autre part, nous avons utilisé des spécifications dans la fonction \texttt{find} de \texttt{MongoDB} pour afficher la fiche de Jacques WEBER et pour filtrer selon une compagnie spécifique.

\end{document}
